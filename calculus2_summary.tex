\documentclass[unfonts,oneside,a4paper]{oblivoir}

\usepackage{kotex}
\usepackage{microtype}
\usepackage{bm}

\setcounter{secnumdepth}{4}
\renewcommand{\thesubsection}{\arabic{subsection}}

\usepackage{mathpazo}
\setmainfont{TeX Gyre Pagella}
\setmainhangulfont[ItalicFont={*},ItalicFeatures={FakeSlant=.167}]{NanumMyeongjo}
\usepackage{anyfontsize}

\usepackage{amsmath,mathtools,amssymb,amsthm,thmtools}

\theoremstyle{definition}
\newtheorem{definition}{정의}[subsubsection]

\theoremstyle{theorem}
\newtheorem{theorem}{정리}[subsubsection]

\usepackage{diffcoeff}
\diffset[roman = true]

\renewcommand{\vec}[1]{\bm{\mathit{#1}}}
\newcommand{\vecz}{\bm{\mathrm{0}}}
\newcommand{\dd}{\mathrm{d}}
\newcommand{\dD}{\mathrm{D}}

\title{수학 및 연습 2\\내용 요약}

\author{이재호\\\href{mailto:jaeho.lee@snu.ac.kr}{\texttt{jaeho.lee@snu.ac.kr}}}

\date{마지막 수정: \today}

\begin{document}

\maketitle

\setcounter{section}{9}
\section{다변수함수}
\subsection{그래프와 등위면}
\subsubsection{다변수함수}

\begin{definition}
    $n$-공간의 부분집합 $U$에서 정의된 함수
    \begin{equation*}
        f: U \rightarrow \mathbb R
    \end{equation*}
    가 주어졌다고 하자.
    즉, $U$의 각 점 $X = (x_1, \dots, x_n)$에 대하여 실수 $f(X) = f(x_1, \dots, x_n)$이 대응된다고 하자.
    이때 함수 $f$를 $n$변수함수라 하고, $n \geq 2$이면 다변수함수라고도 한다.
\end{definition}

\subsubsection{함수와 그래프}

\begin{definition}
    $n$-공간의 부분집합 $U$에서 정의된 다변수함수 $f$의 그래프는
    \begin{equation*}
        \{(x_1, \dots, x_n, z) \in \mathbb R^{n + 1} \mid (x_1, \dots, x_n) \in U,\ z = f(x_1, \dots, x_n)\}
    \end{equation*}
    으로 정의한다.
\end{definition}

\subsubsection{일차함수의 기울기}

\begin{definition}
    이변수 일차함수 $z = ax + by + c$의 기울기는 기울기 벡터
    \begin{equation*}
        (a, b) = a \vec i + b \vec j
    \end{equation*}
    로 나타낸다.
\end{definition}

\begin{theorem}
    단위벡터 $\vec v = (v_1, \dots, v_n)$에 대하여, 일차함수
    \begin{equation*}
        z = \vec a \cdot \vec x + c = a_1 x_1 + \dots + a_n x_n + c
    \end{equation*}
    의 $\vec v$-방향 기울기는
    \begin{equation*}
        \vec a \cdot \vec v = a_1 v_1 + \dots + a_n v_n
    \end{equation*}
    이다.
\end{theorem}

\begin{proof}
    일차함수 $z$의 $\vec v$-방향 기울기는
    \begin{equation*}
        \frac{z(\vec x + t \vec v) - z(\vec x)}{t} = \frac{(\vec a \cdot (\vec x + t \vec v) + c) - (\vec a \cdot \vec x + c)}{t} = \vec a \cdot \vec v
    \end{equation*}
    이다.
\end{proof}

\subsubsection{등위면}

\begin{definition}
    $n$-공간의 한 부분집합 $U$에서 정의된 다변수함수 $f: U \rightarrow \mathbb R$와 실수 $c$에 대하여, $f$에 대한 $c$의 역상(inverse image)
    \begin{equation*}
        f^{-1}(c) := \{X \in U \mid f(X) = c\}
    \end{equation*}
    를 $f$의 $c$-등위면(level surface) 또는 $c$-등가면이라고 부른다.
    즉, 등위면은 함숫값이 같은 점들의 모임이다.
\end{definition}

\subsection{연속함수}

\begin{definition}
    $n$-공간의 부분집합 $U$에서 정의된 함수 $f: U \rightarrow \mathbb R$와 $U$의 한 점 $P$에 대하여
    \begin{equation*}
        \lim_{X \rightarrow P} f(X) = f(P)
    \end{equation*}
    일 때, $f$는 점 $P$에서 연속이라고 정의한다.
    달리 쓰면,
    \begin{equation*}
        \lim_{X \rightarrow P} |f(X) - f(P)| = 0
    \end{equation*}
    이다.
\end{definition}

\subsubsection{연속함수}

\begin{definition}
    함수 $f: U \rightarrow \mathbb R$가 정의역 $U$의 모든 점에서 연속이면, $f$를 연속함수라고 한다.
\end{definition}

\subsubsection{최대최소 정리}
\paragraph{유계인 닫힌 집합}

\begin{definition}
    양수 $r$에 대하여 $n$-공간의 한 점 $P$를 중심으로 하고 반지름의 길이가 $r$인 열린 공(open ball)은
    \begin{equation*}
        \mathbb B^n (P, r) := \{X \in \mathbb R^n \mid |X - P| < r\}
    \end{equation*}
    를 뜻한다. 또  $n$-공간의 한 점 $P$를 중심으로 하고 반지름의 길이가 $r$인 닫힌 공(closed ball)은
    \begin{equation*}
        \overline{\mathbb B}^n (P, r) := \{X \in \mathbb R^n \mid |X - P| \leq r\}
    \end{equation*}
    를 뜻한다.
    
    열린 공 또는 닫힌 공을 그냥 공이라고 하고, 1차원 공은 구간, 2차원 공은 원판이다.
\end{definition}

\begin{definition}
    $n$-공간의 부분 집합 $U$가 유계(bounded)라는 것은 $U$를 포함하는 공이 존재한다는 뜻이다.
    이는 함수
    \begin{equation*}
        U \rightarrow \mathbb R,\quad X \mapsto |X|
    \end{equation*}
    가 유계 함수라는 것과 동치이다.
\end{definition}

\begin{definition}
    $n$-공간에서 경계를 포함하는 부분집합을 닫힌 집합이라고 부른다.
\end{definition}

\begin{theorem} [최대최소 정리]
    $n$-공간의 유계인 닫힌 집합에서 정의된 연속함수는 최댓값과 최솟값을 가진다.
\end{theorem}

\subsection{방향미분과 편미분}
\begin{definition}
    $n$-공간의 열린 집합 $U$에서 정의된 함수 $f: U \rightarrow \mathbb R$와 $U$의 한 점 $P$ 및 벡터 $\vec v$에 대하여, 극한값
    \begin{equation*}
        \dD_{\vec v} f(P) := \lim_{t \rightarrow 0} \frac{f(P + t \vec v) - f(P)}{t} = \diff*{}{t}{0} f(P + t \vec v)
    \end{equation*}
    가 존재하면 이 값을 점 $P$에서 $f$의 $\vec v$-방향미분계수 또는 $\vec v$-방향 순간변화율이라고 한다.
    단위벡터 $\vec v$에 대하여 $\vec v$-방향미분계수의 기하학적 의미는 함수 $f$의 그래프를 직선 $P + t \vec v$ 위에 한정하여 그렸을 때, 점 $(P, f(P))$에서 $f$의 그래프의 기울기를 뜻한다.
\end{definition}
\end{document}
